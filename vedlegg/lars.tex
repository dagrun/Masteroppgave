\chapter{Lars 57 år} \label{chap:lars}

Lars er 57 år og jobber som mellomleder i en større bedrift. Han er gift og har tre barn. Lars er travelt opptatt med jobb, trener mye og har akkurat kjøpt seg hytte på fjellet som han pusser opp og utvider i ledige stunder. Lars og kona har tre engelsksettere som de går på fuglejakt med om høsten, men som ellers stort sett er plassert i kennel. 

Lars har alltid vært i god form og aldri brukt legemidler fast unntatt Nicorette da han sluttet å røyke for noen år siden. Han har brukt mye kosttilskudd tidligere, men har sluttet med det etter at han måtte begynne med blodfortynnende medisin.

Lars har nettopp gjennomgått et større hjerteinfarkt og føler seg ganske redusert fortsatt. Han fikk også konstatert hjerteflimmer. Lars kjenner seg nedfor og kona som er sykepleier lurer på om han kan ha fått depresjon som en bivirkning av blodtrykksmedisinen han har begynt med. Han er også plaget med impotens og mareritt.

Legen på sykehuset var svært tydelig da han snakket med Lars og ga beskjed om at han måtte bruke flere legemidler resten av livet for å hindre nye hjerteinfarkt. Lars, som ellers er vant til å lede diskusjoner og være den som gir andre ordre, ble litt «satt ut» og fikk seg ikke til å stille spørsmålstegn ved noe av det legen sa eller å spørre om utdypende informasjon om legemidlene han fikk skrevet ut.  

Lars har forsøkt å lete seg frem til informasjon om legemidlene sine på nettet, men kona sier han ikke må tro på mesteparten av det han finner, det er bare skremselspropaganda. Hun leser høyt for ham fra Felleskatalogen uten at det gir ham spesielt mye mer forståelse. 

\textbf{Legemidler:}
\begin{itemize}
\item Marevan tabletter 2,5 mg Tas etter egen liste BLODFORTYNNENDE
\item Selo-zok tabletter 100 mg 1 daglig BLODTRYKKSSENKENDE
\item Renitec tabletter 10 mg 1 daglig BLODTRYKKSSENKENDE
\item Simvastatin tabletter 40 mg 1 kveld KOLESTEROLSENKENDE
\end{itemize}

Lars får standardbehandling for å forebygge nye hjerteinfarkt og annen hjerte-kar-sykdom. Både depresjon, impotens og mareritt kan være bivirkninger av legemidlene han får, men kan også være en følge av hjertesykdommen i seg selv. God oppfølging videre med hensyn på effekter og bivirkninger er viktig
