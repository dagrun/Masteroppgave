\chapter{Historien om Kåre} \label{chap:kaareHistorie}

Legen til Kåre anbefalte han å bruke Mine Medisiner for å holde oversikt over legemidlene sine. Dette synes Kåre høres veldig spennende ut, men han er litt redd for at han skal gjøre det feil. Neste gang datteren til Kåre er på besøk med mann og barn, nevner Kåre Mine Medisiner. Svigersønnen tilbyr seg å hjelpe Kåre med å bruke, og bli trygg på systemet. Svigersønnen hjelper Kåre med å opprette profil, og legge inn informasjon om sine legemidler. Før svigersønnen drar igjen legger han Mine Medisiner sin nettside som bokmerke på Kåre sin datamaskin, og gir han en kjapp innføring i bruk av systemet. 

Kåre har mye tillit til Mine Medisiner ettersom det er noe legen har anbefalt. 

Hver søndag lager kona til Kåre, Inger, dosett til han og seg selv. Kåre har veldig mange legemidler, og hun har noen selv også. Hun trenger hjelp med å huske alle legemidlene. En annen ting er at Inger i det siste har begynt å glemme. Det gjør henne enda mer redd for å legge feil legemidler i dosetten. Inger synes Mine Medisiner er veldig god hjelp til å huske alle legemidlene. Hun lager alltid Kåre sin dosett klar før hun begynner på sin egen. Hun logger derfor inn i Mine Medisiner med Kåre sin informasjon, som han har gitt henne. Her trykker hun seg frem til legemiddellisten hans, som viser en fullstendig oversikt over doseringen til hvert av legemiddlene. Ut i fra denne informasjonen klarer Inger å lage dosett til Kåre korrekt. 

Det er tidlig onsdag morgen da Kåre og konen våkner. Kåre gnir søvnen ut av øynene, og reiser seg sakte opp av sengen. Han setter på kaffetrakteren, slik han gjør hver morgen. Mens kaffen koker, gjør konen klar legemidlene til Kåre. Hun tar 12 tabletter ut av dosetten til Kåre, og legger dem på frokostbordet. Når kaffen er klar skjenker Kåre i to store kaffekrus til seg selv og konen. Konen har smurt skiver med hvitost til dem begge. Før Kåre spiser skivene tar han alle tablettene i munnen, tygger dem og skyller dem ned med en stor kopp kald kaffe. Kåre synes det er fryktelig vanskelig å svelge alle tablettene. 

Etter Kåre har svelget tablettene logger han seg på Mine Medisiner for å registrere at han har tatt alle medisinene sine den morgenen. Inne i Mine Medisiner kommer han over informasjon om et av legemidlene hans hvor det er forklart hvordan Kåre skal ta tablettene sine. Der ser at det ikke er lurt å ta alle tablettene samtidig, og at det er lurt å skylle ned med vann. 

Kåre har enorm respekt for leger, og ønsker aldri å være kritisk til det legen sier. Kåre er imidlertid ganske usikker på hvorfor han tar legemidlene sine. En søndag, når Kåre er litt mer nedstemt enn vanlig, bestemmer han seg for å bruke Mine Medisiner til å sjekke hvorfor han tar hvert enkelt legemiddel. Han logger på Mine Medisiner, og velger de legemidlene han er mest usikker på for å finne ut hva de egentlig behandler. Videre ønsker han å se på bivirkningene til hvert enkelt legemiddel, og undersøker også dette når han først er pålogget. 

I den siste tiden har Kåre slitt veldig med fordøyelsen og vært forstoppet. Nå har han kommet inn på tanken om at dette kan skyldes en av tablettene han tar hver morgen. Han vurderer i den sammenheng å droppe og ta den tabletten som medfører ubehaget. Men han vet altså ikke hvilket legemiddel som har skylden. Han logger derfor på Mine Medisiner for å finne ut av dette. Her oppdager han at en av bivirkningene ved bruk av Paralgin Forte er forstoppelse. Han leser advarselen til Mine Medisiner om at ingen beslutninger om endring av legemiddellisten bør tas uten å konsultere lege. Basert på dette bestemmer Kåre seg for å nevne situasjonen med Paralgin Forte og bivirkninger til legen ved neste besøk. 

En dag Kåre bruker Mine Medisiner for å undersøke bivirkninger og interaksjoner oppdager Kåre at Voltaren og Albyl-E sammen gir økt blødningsfare. Kåre bestemmer seg for at dette vil han spørre legen om. 

Neste gang Kåre går til legen manner han seg opp, og nevner at han har sett at Voltaren og Albyl-E sammen gir økt blødningsfare. Legen forteller Kåre at han egentlig ikke bør ta Voltaren, men at han heller bør ta en høyere dose Pinex. Under samme legetime nevner Kåre også informasjonen Mine Medisiner har gitt han om Paralgin Forte, og forstoppelsen bruken har ført til. Legen forklarer Kåre at han kan trygt droppe å ta Paralgin Forte fordi han står på Ibux. Neste gang Kåre logger på Mine Medisiner fjerner Kåre Voltaren og Paralgin Forte fra listen over legemidler i systemet. Kåre endrer også dosen som er lagret for Pinex, og legger inn en kommentar om endringen han har gjort.