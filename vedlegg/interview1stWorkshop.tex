\chapter{Intervju med sykehusfarmasøyter} \label{chap:interview1stWorkshop}

\textit{Svarene på intervjuet blir presentert spørsmål for spørsmål. Svarene fra de to farmasøytene er kombinert og presentert generelt, fordi innholdet er viktigere enn hvem som sa hva. Noen svar er gjengitt som punktlister, med en forklaring til hver punkt. Dette for å få bedre oversikt over innholdet.} 

\textbf{Spørsmål 1: Hvilken informasjon om legemidler synes du det er viktig at pasienter får?}\\
Svarene kan deles inn i følgende kategorier:
\begin{itemize}
\item Årsak til å ta legemiddelet
\item Dose og varighet på behandling
\item Praktisk bruk
\item Bivirkninger
\end{itemize}


\textit{Årsak til å ta legemiddelet:} \\
Farmasøytene var enige om at den viktigste informasjonen for en pasient er hvorfor han tar legemiddelet. Denne informasjonen blir gitt av legen, ofte muntlig. For en pasient med mange legemidler kan det være utfordrende å huske all informasjonen legen gir. I noen tilfeller blir informasjonen skrevet på forpakningen til legemiddelet også. 

Farmasøytene hadde opplevd at pasienter vurderte å slutte å ta sine legemidler, fordi vedkommende ikke husket årsaken til å ta det. Behovet for informasjon fra forskrivende lege viser at en nyttig funksjonalitet for Mine Medisiner kan være å knyttes til eksterne systemer for å hente informasjon om forskrivninger fra en sikker kilde. 

\textit{Dose og varighet:} \\
Det samme legemiddelet kan bli gitt i forskjellige doser til forskjellige pasienter, for å behandle forskjellige symptomer. Varigheten på behandlingen varierer også fra pasient til pasient. Denne individuelle informasjonen kan være viktig i et system som Mine Medisiner, fordi informasjonen ikke direkte kan søkes opp på Internett eller i pakningsvedlegget. Denne informasjonen blir skrevet på forpakningen og/ellers gis muntlig til pasienten, i likhet med årsaken til å ta legemiddelet. Dette underbygger forslaget om legen burde ha tilgang på Mine Medisiner, og kunne legge til pasientspesifikk informasjon i systemet.

\textit{Praktisk bruk:} \\
Sykehusfarmasøytene legger mye vekt på den praktiske bruken av legemidler. De opplever ofte at pasienter ikke tar legemidlene slik de skal. Den ene farmasøyten forteller en historie om en pasient som ble gitt en allergi spray fordi hun var blitt allergis for katten sin. Senere hadde den samme pasienten kommet tilbake på apoteket fordi hun var misfornøyd med effekten til allergi sprayen. Pasienten forklarte at hun hadde sprayet katten hver dag, men at hun fortsatt ble plaget av allergiske reaksjoner. Pasienten hadde altså misforstått og trodde at sprayen var til katten og ikke til pasienten selv. Historien støtter teorien om at pasienter kan misforså hvordan legemidler skal brukes. 

Farmasøytene forklarer at på \href{www.felleskatalogen.no} finnes det kortfilmer som forklarer hvordan noen legemidler skal brukes. Disse filmene er tiltenkt pasietner som bruker ikke-perorale legemidler, feks sprayer og inhalatorer. Selv om disse filmene kun gjelder for ikke-perorale legemidler anser sykehusfarmasøytene praktisk bruk av legemidler som viktig informasjon å ha i et personlig legemiddelinformasjonssystem. Noen legemidler har restriksjoner på når og hvordan de skal tas, for eksempel skal noen legemidler tas utenom måltid med rikelig med vann. Farmasøytene har opplevd at flere pasienter ikke er klar over dette. Dette støtter påstanden om at et personlig legemiddelinformasjonssystem burde inneholde informasjon om praktisk bruk.

\textit{Bivirkninger} \\
Bivirkninger ble nevnt av sykehusfarmasøytene, men de klarte ikke bli enige om alle potensielle bivirkningerbør være synlig  eller ikke. Noen pasienter synes denne informasjonen er nyttig, mens for andre kan det føre til at de slutter å ta legemidlene sine enten fordi de er redd for å få bivirkningene eller pga noceboeffekten\footnote{Nocebo (motsatt av placebo) er det at negative forventninger fører til redusert effekt av en behandling.}. Sykehusfarmasøyten foreslår at det kan være nyttig å kun vise de mest vanlige bivirkningene, med mulighet for å skjule de, kanskje er en god løsning. 

\textbf{Spørsmål 2: Hvor finner pasienter informasjon om sine legemidler?} \\
Sykehusfarmasøytene svarer at de mest tilgjengelige kildene hvor pasienten kan finne informasjon er hos sin doktor, på legemiddelpakningen og i pakningsvedlegget. De nevnte også at det finnes flere nettbaserte informasjonskilder, slik som legemiddelhåndboken og felleskatalogen. De påstår at unge personer er de som benytter seg av Internett som informasjonskilder, mens eldre ofte er mer avhengig av legen sin eller farmasøyter. Venner og familie er nevnes også som en kilde til informasjon.

\textbf{Spørsmål 3:  Hvilken informasjon har pasienter tilgang til i dag?} \\
Sykehusfarmasøytene forteller at hvis pasienten ønsker det kan han klare å finne den informasjonen han søker på nett. Men de antar at noe av informasjonen er vanskelig å forstå uten den faglige bakrunnen som de har.

\textbf{Spørsmål 4: Hvilken informasjon om legemidler har pasienter for lite tilgang til i dag?} \\
Svarene kan deles inn i følgende kategorier:
\begin{itemize}
    \item generiske legemidler
    \item bilde av legemidler
    \item helsekost
    \item har glemt å ta et legemiddel
\end{itemize}

\textit{Generiske legemidler} \\
Informasjon om generiske legemidler kan være vanskelig informasjon for en pasient å finne og forstå. Sykehusfarmasøytene forteller at informasjon om generiske legemidler er tilgjengelig for pasienter, men pasientene synes ofte informasjonen er vanskelig å forstå. Det hender ofte at en pasient som blir forskrevet et generisk legemiddel tar dobbelt så mange legemidler enn det som er nødvendig. Dette problemet oppstår fordi legen eller farmasøyter glemmer å fortelle pasienten at det nye legemiddelet er ekvivalent med det gamle, og erstatter derfor det gamle.

\textit{Bilde av legemiddel} \\
Det kan være vanskelig å finne informasjon om utseende til et legemiddel. For pasienter som tar flere legemidler kan det være vanskelig å vite hvilket legemiddel som er hvilket, spesielt for pasienter som bruker dosett. I tillegg er det mange legemidler som ser like ut. Sykehusfarmasøytene mener at et legemiddelinformasjonssystem kan ha nytte av å vise et bilde av legemidlene til brukeren av et slikt system.

\textit{Helsekost} \\
Helskekost kan ha interaksjoner med legemidler. Dette er det mange pasienter som ikke vet noe om, og derfor oppsøker de heller ikke denne informasjonen. Et legemiddelinformasjonssystem burde i tillegg til å vise legemiddelinteraksjoner også ha støtte for å legge inn helsekost produkter og vise mulige helsekost-legemiddel-interaksjoner.  

\textit{Har glemt å ta et legemiddel} \\
Sykehusfarmasøytene forteller at de har sett flere tilfeller hvor pasienter ikke klarer å finne informasjon om hva pasienten skal gjøre hvis han glemmer å ta et legemiddel en da, ikke føler noe effekt av et legemiddel, eller opplever bivirkninger av et legemiddel. Denne informasjonen finnes i pakningsvedleggene til legemidlene. Sykehusfarmasøytene antar at det kanskje er noe med presantasjonsmåten eller språket som gjør at pasienten ikke finner det eller ikke forstå det som står der. Pasienter med dosett har kanskje ikke pakningsvedlegg tilgjengelig, så for dem er det ekstra vanskelig å finne denne informasjonen.

\textbf{Spørsmål 5 Hvilken måte mener du er best for å formidle legemiddelinformasjon tilpasienter?} \\
Man bør bare formidle informasjon som er relevant for den enkelte pasient. 

Pasientene bør få nok tid til å lese og prosessere informasjon som blir gitt til dem. 

\textbf{Spørsmål 6: Hvilke faktorer påvirker hvorvidt legemiddelinformasjon blir fortstått av pasientene?} \\
Den viktigste faktoren er hvilket språk man bruker for å formidle informasjonen. Språket bør være uformelt og lett å forstå. Det er imidlertid viktig at pasientene ikke synes man ser ned på dem, og snakker til dem som om de er barn. 

