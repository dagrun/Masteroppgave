\chapter{Klara 87 år} \label{chap:klara}
Klara er en enke på 87 år. Hun har nettopp flyttet inn i en nybygd omsorgsbolig, sentralt i en større norsk by, fordi hun er blitt svært dårlig til bens og har falt flere ganger. Klara er klar i hodet og interessert både i menneskene rundt seg og samfunnet generelt. Hun savner huset og spesielt hagen hun måtte flytte fra, men har god innsikt i sin egen situasjon og skjønner at det er best for både henne og de pårørende at hun har flyttet. 

Klara har hjelp til å stå opp og legge seg, noe matlaging og renhold, men håndterer legemidlene sine selv. Hun har i utgangspunktet stor tiltro til autoriteter, deriblant fastlegen sin og andre leger. 

Datteren hennes bor i samme by og er ofte innom på besøk. Datteren har det moren kaller ``et alternativt livssyn'', er kritisk til skolemedisin og synes moren tar alt for mange tabletter. Datteren synes også at moren har for stor tiltro til leger og at hun er litt ``godtroende'' som tar så mange legemidler uten egentlig å vite hverken hvorfor hun tar dem eller hvilke bivirkninger de kan ha. Klara har til nå vært mildt overbærende til datterens kritiske spørsmål, men etter en sykehusinnleggelse på grunn av et lårhalsbrudd, har hun nå fått enda flere legemidler og synes også selv det begynner «å bli litt mye», som hun sier. 

Klaras ene dattersønn har overtalt henne til å kjøpe seg en iPad til å se på film og nettTV og generelt holde seg orientert. Han holder også på å lære henne å bruke Facetime og Skype for at hun skal kunne ha kontakt med andre barn og barnebarn som bor lenger unna. Dette synes Klara er stor stas og hun synes iPad’en er en fantastisk oppfinnelse!

Klara er operert for lårhalsbrudd for fire uker siden. Kommer seg fint og har lite smerter. Hun har vært plaget med urinveisinfeksjoner (blærekatarr) og får medisiner for å hindre at hun skal plages med dette. På sykehuset synes Klara det kunne være litt vanskelig å sovne om kvelden, men nå sover hun godt. 

\textbf{Legemidler:}
\begin{itemize}
\item Imovane tabletter 7,5 mg 1 om kvelden SOVEMEDISIN 
\item Albyl-E tabletter 75  mg 1 daglig BLODFORTYNNENDE
\item Persantin retard kapsler 200 mg 1 morgen og 1 kveld BLODFORTYNNENDE
\item Simvastatin tabletter 20 mg 1 om kvelden KOLESTEROLSENKENDE
\item Renitec tabletter 5 mg 1 daglig BLODTRYKKSSENKENDE
\item Vagifem vaginaltabletter 10 mikrogram 1 tablett i skjeden annenhver dag ØSTROGENBEHANDLING
\item Paralgin forte tabletter 1-2 tabletter 3-4 ganger daglig VED SMERTER
\item Ferromax tabletter 65 mg 2 tabletter morgen og 1 tablett kveld JERNTILSKUDD
\item Paracet tabletter 500 mg  1 tablett ved hodepine SMERTESTILLENDE 
\item Voltaren tabletter 50 mg 1 tablett ved giktsmerter SMERTESTILLENDE 
\item Zyrtec tabletter 10 mg 1 daglig ALLERGIMEDISIN
\end{itemize}

Klara hadde et TIA-anfall (lite slag/drypp) for 11 år siden og siden den gang har hun brukt Albyl-E, Persantin, Simvastatin og Renitec for å hindre nye slag. Etter at hun ble utskrevet fra sykehuset, har Klara vært en del kvalm og hun er mye plaget med forstoppelse. Når Klara en sjelden gang har hodepine eller influensa, maks to ganger i måneden, tar hun en Paracet og det synes hun virker bra. Hun plages av og til med litt med giktsmerte og da tar hun Voltaren som hun kjøper på apoteket uten resept, og tar dette nesten hver dag, inntil 2 tabletter daglig. Klara sier hun tar 1 tablett Paralgin Forte fast 3 ganger daglig, uansett om hun har vondt eller ikke

