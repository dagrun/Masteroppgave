\chapter{Kåre 77 år} \label{chap:kaare}
Kåre er et typisk eksempel på en pasient i sin aldersgruppe, med flere vanlige sykdommer. Kåre er ikke helt klar over hvorfor han tar hvert enkelt legemiddel. Han er klar over hva som er galt med han, men han klarer ikke å knytte legemidlene til problemene sine. Kåre tilhører en aldersgruppen med en inngrodd respekt for prester, leger o.s.v. Det er veldig typisk at hvis fastlegen sier at de bør ta et legemiddel, så tørr de ikke diskutere.

Kåre har ikke smarttelefon eller ipad, men han har en pc som svigersønnen har hjulpet han med å sette opp. Han synes det er stas å bruke pc-en.

Kåre bor sammen med kona, i en ny leilighet som går over ett plan, sentralt i byen. Kona legger dosett for dem begge hver søndag. De har to døtre med familie i nærheten, som de har god kontakt med. Kåra og kona får mye praktisk hjelp av dem, men er ellers selvhjulpne.

Kåre drikker endel alkohol, og har hatt diabetes type II i halvannet år. Han er blitt litt dårlig til beins det siste året og kona synes han er ``tung til sinns''. Kåre er enig med kona, som stort sett fører ordet. Begge to er stille og forsikte typer. De oppfattes som å være opptatt av å ``svare korrekt'' om blant annet bruk av legemidler. De ønsker ikke å være til bry, men setter stor pris på informasjon når den tilbys.

Kåre er stomioperert for å få utlagt tarm, for 6 måneder siden. Han var nylig innlagt gastrokirurgisk avdelingen på grunn av akutte magesmerter. Legene mistenker tarmslyng, men Kåre er nå klar for hjemreise uten at det er gjort noe kirurgisk inngrep. Han er i mye bedre form og synes det er OK å bli sendt hjem.

\textbf{Legemiddelallergi:} Får utslett av Fanalgin®

\textbf{Legemidler:}
\begin{itemize}
\item Amaryl 12 mg 1 tablett formiddag MOT DIABETES
\item Cipramil 20 mg 1 tablett formiddag MOT TUNGSINN
\item Paralgin forte 1 tablett ved magesmerter, maks 3 tabletter daglig VED MAGESMERTER
\item Pinex 500 mg 1 tablett mot smerter ved behov SMERTESTILLENDE
\item Betolvex depot injeksjon 1mg hver 3. måned
\item Albyl-E 75 mg 1 tablett formiddag BLODFORTYNNENDE
\item TrioBe 1 tablett formiddag 
\item Voltaren 50 mg 1 tablett 3 ganger daglig SMERTESTILLENDE
\item Omega-3 2 kapsler formiddag
\end{itemize}

Kåre synes det går greit å huske medisinene, når kona minner ham på det. Albyl-E er jo det samme som Dispril, som er smertestillende, så hvis pasienten er i fin form tar han ikke alltid denne. Han tar ut en hvit en, og håper at det er Albyl-E. Ikke sikkert det alltid er akkurat den han lar være å ta. På forespørsel sier Kåre at det er litt mange tabletter å ta om morgenen. Han klarer ikke svelge alle på en gang, men når han tygger ``hele munnfullen'' og skyller ned med en stor kopp kald kaffe. Kåre tror han må ta ALLE tablettene på en gang, det er ca 9 stykk, fordi det står ``morgen'' på alle sammen. Albyl-E skal ikke tygges, og burde ikke tas slik Kåre gjør det. Det kan være lurt å ta tablettene én og én. Tabletter bør helst tas stående eller sittende. Det er ikke lurt å ta tabletter når man ligger. Denne informasjonen om praktisk bruk kan være nyttig for Kåre. 
