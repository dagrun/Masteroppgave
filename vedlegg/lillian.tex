\chapter{Lillian 27 år} \label{chap:lillian}
Lillian er en 27 år gammel kvinne som har hatt epilepsi siden hun var 9 år. Hun er samboer med barndomskjæresten sin på det lille tettstedet hvor hun har vokst opp, og jobber på en Rema-butikk i gangavstand fra leiligheten de bor i. Nå ønsker Lillian og samboeren seg barn. Samboeren vil hun bare skal slutte med p-pillene med en gang, mens Lillian er mer usikker, siden hun har hørt så mange historier om hva som kan gå galt. 

Lillian har brukt epilepsimedisin i mange år og vet at noen epilepsimedisiner kan være fosterskadelige. Legen hennes har tidligere sagt at epilepsi-anfall også kan være skadelige for et eventuelt foster. Lillian er veldig usikker og har stort behov for informasjon. Hun har ingen god relasjon til fastlegen sin og har ikke vært i kontakt med spesialisten på sykehuset på nesten tre år, siden det er så lenge siden hun har hatt anfall. Hun lurer blant annet på om hun skal gjøre endringer i legemiddelbruken sin før hun slutter med p-pillene, om hun skal fortsette med de legemidlene hun har eller begynne med nye, om hun bare skal endre dosen eller om hun helt skal slutte med epilepsimedisiner. Noen har også sagt til henne at alle gravide bør ta folsyre i første del av graviditeten og at dette er spesielt viktig for pasienter med epilepsi. Lillians mor har tidligere brukt et B-vitamin-preparat som heter TrioBe. Det inneholder blant annet folsyre, så Lillian har begynt å bruke dette, selv om hun ikke har sluttet med p-pillene ennå. 

Lillian har prøvd ganske mange forskjellige legemidler mot epilepsi opp gjennom årene, men har nå brukt Keppra og Lamictal de siste tre-fire årene. Legemidlene ser ut til å virke godt, hun har ikke hatt epileptiske anfall på neste tre år og hun opplever ingen plagsomme bivirkninger. Lillian tar blodprøver hos fastlegen to ganger i året for å se at hun får riktig dose av de to legemidlene mot epilepsi. Blodprøvesvarene ligger innenfor det såkalte referanseområdet, det vil si at doseringen er godt tilpasset hennes situasjon. 

\textbf{Legemidler:}
\begin{itemize}
\item Keppra (levetiracetam) tabletter 1500 mg 2 ganger daglig MOT EPILEPSI
\item Lamictal (lamotrigin) tabletter 100 mg 2 ganger daglig MOT EPILEPSI
\item Yasmin tabletter 1 daglig P-PILLER
\item Ibux tabletter 400 mg 1 tablett inntil 3 ganger daglig MOT HODEPINE 
\item TrioBe tabletter 1 daglig B-VITAMINTILSKUDD
\end{itemize}

Lillian tar begge legemidlene mot epilepsi regelmessig og slik hun har fått beskjed om. Hun er opptatt av å ikke få nye anfall, både fordi hun opplever det som ubehagelig, men også fordi hun bor «litt langt fra alt og alle» og er avhengig av å kunne kjøre bil. Legemidler mot epilepsi kan i tillegg være fosterskadelige. Epileptiske anfall kan også være skadelige – både for barnet og for moren. Det er derfor viktig med god informasjon og planlegging når pasienter med epilepsi ønsker å bli gravide. 

Ibux har Lillian fått på resept fra fastlegen sin for snart et år siden. Hun ønsket selv utredning for migrene, men fastlegen mente det i første omgang var nok å gi henne smertestillende uten videre undersøkelser. Lillian føler seg litt avspist av fastlegen og har ikke tatt kontakt med ham mer, selv om hun fortsatt ofte opplever sterke hodepine og tar Ibux nesten hver dag. Bruk av smertestillende legemidler som f.eks. Ibux over lenger tid, kan virke mot sin hensikt og faktisk gi hodepine. En kan ikke si sikkert at det er tilfelle for Lillian, men det kan ikke utelukkes. Bruk av Ibux og lignende legemidler er heller ikke gunstig for kvinner som ønsker å bli gravide elle er gravide. Lillian bør slutte med Ibux. 

TrioBe har Lillian fått av sin mor. Det inneholder riktignok 0,8 mg folsyre, men også to andre B-vitaminer som Lillian i utgangspunktet ikke trenger. Kvinner som bruker den typen legemidler mot epilepsi som Lillian gjør, bør bruke 1 mg folsyre daglig gjennom hele svangerskapet. Lillian bør derfor slutte med TrioBe og begynne med tabletter som inneholder 1 mg folsyre og ikke noe annet.

