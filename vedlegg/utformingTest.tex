\chapter{Utforming av eksperiment} \label{chap:utforming}

\textbf{Del 1 Introduksjon:} \\
Jeg er en del av en gruppe på tre studenter som er i ferd med å skrive mastergrad for institutt for datateknikk og informasjonsteknologi ved NTNU. Oppgaven vår handler om å presentere legemiddelinformasjon til pasienter. 


\underline{Mine medisiner:}\\
I forbindelse med mastergraden vår ønsker vi i dag å utføre et forsøk på et system kalt Mine Medisiner. Hensikten med forsøket er å undersøke systemets måte å presentere informasjon. Dette gjør vi ved at du får en rekke utsagn som du skal si deg enig eller uenig i basert på informasjonen du finner i Mine Medisiner. Dette er på ingen måte en test av deg eller dine kunnskaper, kun en test av systemet. Mine Medisiner er ikke et ferdig system, noe som gjør at det kan oppstå noen feil underveis, at noen knapper ikke fungerer og at systemet er litt tregt. Fordi systemet kan være litt tregt kan det være lurt å prøve ting flere ganger, da det kan hende det fungerer neste gang du prøver. Det er ikke din feil hvis oppgaven er vanskelig å utføre. 

\underline{Pakningsvedlegg:}\\
I forbindelse med mastergraden vår ønsker vi i dag å utføre et forsøk på pakningsvedlegg. Du vil få utlevert pakningsvedleggene skriftlig. Hensikten med forsøket er å undersøke pakningsvedleggenes måte å presentere informasjon. Dette gjør vi ved at du får en rekke utsagn som du skal si deg enig eller uenig i basert på informasjonen du finner i pakningsvedleggene. Dette er på ingen måte en test av deg eller dine kunnskaper, kun en test av pakningsvedleggene. Det er ikke din feil hvis oppgaven er vanskelig å utføre. 

Jeg kommer først til å stille deg noen innledende spørsmål, så gir jeg deg noen utsagn du skal vurdere ved hjelp av systemet, til slutt vil jeg stille deg noen oppsumeringsspørsmål. De som sitter borte i hjørnet skal bare observere.

Det er fint om du sier ifra om det er ting du lurer på eller synes er vanskelig underveis, så kan vi diskutere det etterpå. 

Foran deg ligger det skriveutstyr og papir. Dette kan du bruke til å ta notater underveis dersom du ønsker det. Når du skal bruke Mine Medisiner skal du benytte deg av datamaskinen som står foran deg. På den store skjermen kan vi se hva du gjør på datamaskinen. 

Under selve forsøket kan jeg ikke tilby noe hjelp. Når du føler deg ferdig med en oppgave er det fint om du oppsummerer resultatet ditt høyt. Si ifra når du er klar for å motta neste oppgave. 

Du kan avbryte forsøket når du vil, uten at du trenger å forklare hvorfor du velger å avbryte forsøket.

Det første vi trenger av deg er en underskrift på dette samtykkeskrivet, vedlegg~\ref{chap:samtykke}.

Er det noe du lurer på før vi begynner? 

\textbf{Del 2: Demografiske spørsmål}
Deltakeren begynner med å fylle ut et skjema med demografiske spørsmål, vedlegg~\ref{chap:demografi}.

\textbf{Del 3: Spørsmål for å sjekke kunnskapsnivået til deltakeren:} \\
Vi stiller spørsmålet og noterer svarene uten å oppklare om de har forstått det rett eller ikke. 
\begin{enumerate}
\item Hva forstår du av begrepet interaksjon i forbindelse med legemidler (legemiddelinteraksjon)?
\item Hva forstår du av begrepet bivirkning i forbindelse med legemidler?
\item Har du lest pakningsvedlegg, felleskatalogen, legemiddelhåndboka eller lignenede? Fant du det du lette etter?
\item Har du prøvd å finne ut om bivirkninger til noen legemidler du har tatt?
\item Har du prøvd å finne ut om noen legemidler du har tatt har vært er uheldig å kombinere med hverandre?
\suspend{enumerate}
Vi gir deltakeren legemiddellisten til Håkon, en personas som brukes senere i forsøket, vedlegg~\ref{chap:haakon}
\resume{enumerate}
\item Hva vet du fra før om legemidlene i legemiddellisten?
\item Hva vet du om hvorvidt noen av legemidlene i listen er dårlig å kombinere med hverandre?
\item Hva vet du om bivirkningene til legemidlene i legemiddellisten?
\end{enumerate}

\underline{Introduksjon til Mine Medisiner:}\\
Vi logger inn som Kåre i Mine Medisiner og illustrerer hvordan systemet fungerer og ser ut for en fiktiv person Kåre.

Mine Medisiner er et system hvor man kan legge inn hvilke legemidler man selv tar i en personlig legemiddelliste. Det er også mulig å legge inn dose, når legemiddelet skal tas og hvorfor man tar det. Det er mulig å trykke på hvert enkelt legemiddel i listen for å se mer informasjon, blant annet bivirkninger, hvordan det inntas og et bilde av legemiddelet. 

I tillegg til legemiddellisten har Mine Medisiner to funksjoner til, under hver sin fane: Interaksjonsgraf og Søk i symptomer. Interaksjonsgrafen visualiserer interaksjonene som forekommer mellom legemidler du har lagt inn. Det er mulig å få mer informasjon ved å trykke på elementene i grafen. Det er også mulig å velge å huke av hvilke legemidler som skal vises i grafen. 

Søk i symptomer har et søkefelt, og en klikkbar menneskefigur. Ved å skrive inn et symptom man har opplevd, i tekstfeltet, får man en liste over hvilke av dine legemidler som kan forårsake eller vil kurere symptomet det er søkt på. Dersom man f.eks har opplevd smerte i en spesifikk kroppsdel er det mulig å klikke figuren for å få et søkeresultat med legemidler relatert til den kroppsdelen. 

\underline{Introduksjon til pakningsvedlegg:}\\
Et pakningsvedlegg følger i forpakningen til alle legemidler. Det inneholder informasjon om hva legemiddelet brukes mot, når det ikke skal brukes, bivirkninger, hvordan det brukes, forsiktighetsregler osv. Pakningsvedleggene kan varierer i innhold og oppbygging, men består hovedsaklig av fritekst. 

\textbf{Del 4: Praktisk gjennomføring:}\\
Deltakeren får tid til å sette seg inn i rollen som Håkon, vedlegg~\ref{chap:haakon}. Legemiddellisten til Håkon brukes under gjennomføringen av den praktiske delen. 

Deltakeren får utdelt et sett med utsagn, vedlegg~\ref{chap:paastander}. Vi tar tiden mens deltakeren tar stilling til utsagnene. Når deltakeren har tatt stilling til alle utsagnene og gir oss beskjed om at deltakeren er ferdig blir deltakeren bedt å forklare hvorfor han svarte som han gjorde.

Etter alle utsagnene er gjennomgått blir deltakeren forklart at han ikke lenger Håkon og at resten av spørsmålene skal han besvare ut i fra egne meninger.

\textbf{Del 5: SUS-skjema:}
Deltakeren fyller ut SUS-skjema for systemet som ble benyttet i forsøket, se vedlegg~\ref{chap:SUSPak} for pakningsvedlegg og vedlegg~\ref{chap:SUSMM} for Mine Medisiner.

\textbf{Del 6: Spørsmål knyttet til deltakerens læringsutbytte:}
\begin{enumerate}
\item Hva kan Renitec brukes mot?
\item Hva er mulige bivirkninger av Albyl-E?
\item Hva er mulige bivirkninger av Triatec?
\item Hvilke av legemidlene i legemiddellisten kan det være uheldige å kombinere?
\item Er det uheldig å kombinere Simvastatin og Albyl-E?
\item Er det uheldig å kombinere Triatec og Dispril?
\end{enumerate}

\textbf{Del 7: Reaction card:}\\
Deltakeren får velge fem av ordene i tabell~\ref{tab:reactionCardTabVedlegg} de mener best beskriver systemet som ble benyttet i forsøket. Etter deltakeren har valgt ut 5 ord går vi igjennom hvert ord og lar deltakeren forklare hvorfor han valgte disse ordene.

\begin{table}[H]
    \centering
    \begin{tabular}{ | p{2.6cm} | p{2.6cm} | p{2.6cm} | p{2.6cm} |}
      \hline
        Avansert & Komplekst & Verdifull & Frustrerende \\ \hline
        Forståelig & Forvirrende & Lett å bruke & Vanskelig å bruke \\ \hline
        Effektiv & Spennende & Kjent & Stabil \\ \hline
        Inkonsekvent & Ineffektiv & Meningsfull & Nedlatende\\ \hline
        Skremmende & Intuitiv & For teknisk & Personlig \\ \hline
        Irrelevant & Motiverende & Overveldende & Forutsigbar  \\ \hline
        Enkelt & Stressende & Tidkrevende & Ukonvensjonell \\ \hline
        Treg & Tidsbesparende & Troverdig & Uønsket \\\hline
        Nyttig & Upersonlig & Uforståelig & Relevant \\\hline
        Pålitelig & Tilfredsstillende & Uforutsigbar & \\\hline
        \end{tabular}
    \caption{Utvalget av ord deltakeren kunne velge mellom i den siste del av forsøket}
    \label{tab:reactionCardTabVedlegg}
\end{table}
