\chapter{Håkon 73 år} \label{chap:haakon}

Håkon er 73 år og bor hjemme sammen med sin kone. Begge er ``klare i hodet'', men noe dårlige til beins. Med litt praktisk hjelp til snømåking og helgehandelen fra barnebarnet som bor i nabohuset, klarer de seg selv. 

Håkon har hatt to TIA-anfall/drypp (hjerneslag hvor symptomene trekker seg tilbake ila 24 timer) det siste halvannet året. Blodtrykket er fint, men kolesterolverdiene er noe høye. Håkon bekymrer seg for å få nye TIA-anfall eller hjerneslag.

\textbf{Legemidler:}
\begin{itemize}
\item Albyl-E 75 mg – 1 tablett morgen. BLODFORTYNNENDE. FOREBYGGENDE MOT SLAG
\item Simvastatin Bluefish 40 mg – 1 tablett kveld.  KOLESTEROLSENKENDE. FOREBYGGENDE MOT SLAG
\item Triatec 5 mg – 1 tablett morgen.  BLODTRYKKSSENKENDE. FOREBYGGENDE MOT SLAG
\item Ibux tabletter 200 mg SMERTESTILLENDE
\item Dispril oppløselige tabletter 300 mg  SMERTESTILLENDE
\end{itemize}

Engasjert kone: ``Vi tar ALLE medisinene slik vi skal, begge to!'' 

Begge ektefellene bruker Ibux og noen ganger noe gammel Dispril de har liggende når de har hodepine eller andre smerter. Det har de alltid gjort. Styrken på Ibux er 200 mg. De tar en eller to tabletter når de har vondt. Noen ganger én gang daglig, andre ganger to eller tre ganger daglig. Det tas ikke fast, bare ved behov. Håkon mener han kanskje tar fire-fem tabletter i uken i gjennomsnitt. Dispril oppløselige tabletter har en styrke på 300 mg og det tar Håkon når han ikke har tid til at vanlige tabletter som Ibux skal virke. Han anslår at han bruker Dispril 1-2 ganger i uka, aldri mer enn 1 tablett daglig.

Albyl-E (blodfortynnende), Simvastatin (kolesterolsenkende) og Triatec (mot høyt blodtrykk) er eksempel på standard behandling etter TIA for å forhindre nye slag.  

Håkon har brukt Albyl-E og Triatec i angitte dosering siden første TIA-anfall for halvannet år siden. Simvastatin ble startet opp samtidig, men da i 20 mg. Etter andre TIA-anfall for ni måneder siden ble Simvastatin-dosen økt til 40 mg.
