\chapter{Konklusjon} \label{chap:konklusjon}
I denne masteroppgaven ble det utviklet en prototype som inneholdt en visuell fremstilling av personlig legemiddelinformasjon. Denne ble sammenlignet med pakningsvedlegg i et eksperiment. 

Før vi utførte eksperimentet hadde vi følgende hypoteser:
\begin{enumerate}
 \item Ved å bruke prototypen får pasienter raskere svar på spesifikke spørsmål enn ved å bruke pakningsvedlegg.
 \item Ved å bruke prototypen oppfatter pasienter svaret på spesifikke spørsmål mer korrekt enn ved å bruke pakningsvedlegg.
 \item Ved å bruke pakningsvedlegg tilegner pasienter seg kunnskap utover det de leter etter, i større grad enn ved å bruke prototypen
\end{enumerate}

Resultatene av forsøkene viste at deltakerene raskere fant informasjonen de lette etter ved å bruke prototypen enn ved å bruke pakningsvedlegg. Tilbakemeldingene som kom i Reaction Card-delen av forsøkene fortalte at deltakerene opplevde pakningsvedlegg som tidkrevende og inneffektive, mens de opplevde prototypen som tidsbesparende. I tilbakemeldingene forklarte deltakerene at de trodde det tok lang tid å bruke pakningsvedlegg fordi det var så store mengder informasjon som presenteres i dem. Det tok lang tid å få oversikt over informasjonen og finne det de lette etter. 

Deltakerene svarte mer korrekt på påstander ved å bruke prototypen enn ved å bruke pakningsvedlegg. Når det gjaldt bivirkninger var det flere av deltakerene som ikke fant den informasjonen de trengte i pakningsvedleggene. Det kan skyldes at informasjonen var gjemt bort i mye annen informasjon. Når det gjaldt interaksjoner var det flere av deltakerene som fant informasjonen de trengte i pakningsvedleggene, men som allikevel ikke svarte korrekt på påstandene. Det kan virke som om deltakerene ikke forstod informasjonen som ble presentert. I tilbakemeldingene på Reaction Card delen av forsøkene forklarte deltakerene dette med at pakningsvedleggene brukte komplisert medisinsk terminologi som de ikke forstod. 

Deltakerene var i større grad i stand til å tilegne seg kunnskap utover det de lette etter ved bruk av pakningsvedlegg enn ved bruk av prototypen. Deltakerene var imidlertid i stand til å huske mer av det de lette etter ved bruk av prototypen.


Forskningsspørsmålet for masteroppgaven var: 
\thesisRQ

Prototypen gav pasienter både raskere og mer korrekt kunnskap om interaksjoner og bivirkninger knyttet til egne legemidler enn pakningsvedlegg. Altså kan en visuell fremstilling av personlig legemiddelinformasjon på et digitalt format gi raskere og mer korrekt kunnskap om bivirkninger og interaksjoner enn tekst fra pakningsvedlegg. 

Vi vet imidlertid lite om hva som gjorde at prototypene gav raskere og mer korrekt kunnskap enn pakningsvedlegg. Eksperimentet klarte ikke å vise at tekst gav dårligere eller bedre resultater enn andre alternativer slik som graf og søkefunskjonalitet. Grunnen til at det ikke kan trekkes noen konklusjon om dette er at det var andre faktorer som også skilte systemene. Tungt språk, mye informasjon og liten grad av personlig tilpasning ser ut til å være viktigere faktorer til at pakningsvedlegg kom dårligere ut i eksperimentet.

Tilbakemeldingene fra deltakerene tyder på at prototypen er mer brukbar enn pakningsvedleggene. Høyere brukbarhet kan føre til at pasienter tilegner seg med kunnskap om egne legemidler. 