\newglossaryentry{legemiddel}
{
    name=Legemidler,
    description={er, i henhold til legemiddelloven § 2, “(...)stoffer, droger og preparater som er bestemt til eller utgis for å brukes til å forebygge, lege eller lindre sykdom, sykdomssymptomer eller smerter, påvirke fysiologiske funksjoner hos mennesker eller dyr, eller til ved innvortes eller utvortes bruk å påvise sykdom”}
}

\newglossaryentry{symptom}
{
    name=Symptom,
    description={er en “(...)subjektiv opplevelse av at noe er unormalt med en selv” \citep{LeksikonSymptom}. Et symptom kan være et av tegnene på en sykdom. Hodepine er et eksempel på et symptom}
}
\newglossaryentry{tegn}
{
    name=Tegn,
    description={er observerbare eller registrerbare symptomer. Tegn kan være synlige eller kan oppdages ved hjelp av medisinske tester. Høyt blodtrykk er et eksempel på et tegn}
}
\newglossaryentry{indikasjon}
{
    name=Indikasjon,
    description={er årsaken til å ta et legemiddel. Et legemiddel kan ha flere indikasjoner. Diabetes type 2 er et eksempel på en indikasjon for Amaryl}
}
\newglossaryentry{generiske legemidler}
{
    name=Generiske legemidler,
    description={er medisinsk likeverdige legemidler fordi de inneholder samme virkestoff i samme styrke. Legemidlene kan ha forskjellige handelsnavn, hjelpestoffer, farge, form og forpakning}
}
\newglossaryentry{effekt}
{
    name=Effekt,
    description={er virkningen et legemiddel har på pasienten som tar legemiddelet}
}
\newglossaryentry{bivirkning}
{
    name=Bivirkning,
    description={er en ikke-tilsiktede effekt et legemiddel kan ha}
}
\newglossaryentry{interaksjon}
{
    name=Interaksjon,
    description={er at legemidler, legemidler og helsekostprodukter, legemidler og mat, eller legemidler og drikke, påvirker hverandre}
}
\newglossaryentry{atc-def}
{
    name=Anatomical Therapeutic Chemical,
    description={er en klassifisering av legemidler ut i fra hvilket organ de virker på, og ut i fra legemidlets kjemiske, farmakologiske og terapeutiske karakteristikk}
}
\newglossaryentry{ICD-def}
{
    name=ICD,
    description={er koder for klassifisering av sykdommer og beslektede helseproblemer}
}
\newglossaryentry{mengde}
{
    name=Mengde,
    description={er antall enheter per inntak av et legemiddel}
}
\newglossaryentry{Styrke}
{
    name=Styrke,
    description={av et legemiddel er innholdet av virkestoff per enhet}
}
\newglossaryentry{Dose}
{
    name=Dose,
    description={er det totale innholdet av virkestoff per inntak av et legemiddel. \(Dose = mengde \times styrke\). En pasient som tar 2 tabletter(mengde) av et legemiddel med styrke på 50 mg tar en dose på 100 mg} 
}
\newglossaryentry{diagnose}
{
    name=Diagnose,
    description={er en sykdoms art og navn. Diagnose stilles på grunnlag av opplysninger om pasienten og av de funn gjort ved en kliniske undersøkelse} 
}
\newglossaryentry{legemiddelgjennomgang}
{
    name=Legemiddelgjennomgang,
    description={er en “strukturert/systematisk evaluering av den enkelte pasientens legemiddelregime i den hensikt å optimalisere effekten av legemidlene og redusere risiko ved legemiddelbruk” \citep{legemiddelgjennomgang}} 
}
\newglossaryentry{elektronisk pasientjournal}
{
    name=Elektronisk pasientjournal,
    description={er et datasystem som inneholder alle opplysninger tilgjengelig i en pasientjournal. En elektronisk pasientjournal inneholder helseopplysninger, prøvesvar, spesialundersøkelser, arbeidsdokumenter og øvrige dokumenter knyttet til pasienten. Elektroniske pasientjournaler kan også inneholde informasjon fylt inn av pasienten selv} 
}
\newglossaryentry{elektronisk resept (e-resept)}
{
    name=Elektronisk resept (e-resept),
    description={er en resept som ikke skrives ut på papir, men som en fysisk eller juridisk person med rett til å rekvirere legemidler sender elektronisk til en sentral reseptdatabase} 
}
\newglossaryentry{etterlevelse}
{
    name=Etterlevelse,
    description={er et begrep som brukes for å beskrive hvorvidt legemidler tas som foreskrevet eller ikke } 
}
\newglossaryentry{fastlege}
{
    name=Fastlege,
    description={er en allmennpraktiserende lege med formell avtale med kommunen om et varig forhold mellom legen og pasientene. Fast legen skal tilby alle pasientene på sin liste alle typer “allmennlegeoppgaver”}
}
\newglossaryentry{helseopplysninger}
{
    name=Helseopplysninger,
    description={er “... taushetsbelagte opplysninger i henhold til helsepersonelloven § 21 og andre opplysninger og vurderinger om helseforhold eller av betydning for helseforhold, som kan knyttes til en enkeltperson”}
}
\newglossaryentry{ordinering}
{
    name=Ordinering,
    description={er at legen bestemmer bruk av legemiddel og dosering, og journalfører dette, jf. forskrift om legemiddelhåndtering § 3 bokstav g}
}
\newglossaryentry{ordinering}
{
    name=Ordinering,
    description={er at legen bestemmer bruk av legemiddel og dosering, og journalfører dette, jf. forskrift om legemiddelhåndtering § 3 bokstav g}
}
\newglossaryentry{pakningsvedlegg}
{
    name=Pakningsvedlegg,
    description={er, i henhold til Legemiddelforskriften § 3-26, “...det vedlegg med opplysninger til brukeren som følger med legemidlet”}
}
\newglossaryentry{personvern}
{
    name=Personvern,
    description={er en samlebetegnelse for rettsregler som tar sikte på ivaretakelse av den personlige integritet; ivaretakelse
av enkeltindividers mulighet for privatliv, selvbestemmelse (autonomi) og selvutfoldelse}
}
\newglossaryentry{rekvirere}
{
    name=Rekvirering (rekvirere),
    description={er, i hendhold til Forskrift om legemidler fra apotek § 1-3 bokstav, en "muntlig, skriftlig eller elektronisk bestilling av legemiddel ved resept eller rekvisisjon"}
}







 
 




