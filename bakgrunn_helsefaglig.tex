\chapter{Legemidler} \label{chap:bakgrunn_helsefaglig}
Legemidler kan ha flere effekter. Et legemiddel har ønsket effekt hvis det har antatt virkning. En ønsket effekt for et legemiddel kan for eksempel være å kurere hodepine. Effekter kan også være negative, og inkluderer bivirkninger og interaksjoner.

Dette kapittelet vil gi en kort introduksjon til begrepene interaksjon og bivirkning som er nødvendig for å forstå forskningsspørsmålet, jf. delkapittel~\ref{sec:forskningssporsmaal}.

\section{Interaksjoner}
En interaksjon er at legemidler, legemidler og helsekostprodukter, legemidler og mat, eller legemidler og drikke, påvirker hverandre \citep{Stockley}. Interaksjoner er delt i to hovedtyper: Farmakokinetiske og farmakodynamiske interaksjoner. 

Farmakokinetiske interaksjoner betyr at et legemiddel forandrer effekten til et annet legemiddel slik at konsentrasjonen av virkestoffet i kroppen endres. Disse interaksjonene kan endre kroppens evne til opptak av legemiddelets virkestoff, distribusjonen av legemiddelet, nyttiggjøring av virkestoffet og eliminering av avfall. 

Farmakodynamiske interaksjoner forekommer når et legemiddel påvirker effekten av et annet legemiddel uten å endre konsentrasjonen i kroppen. Farmakodynamiske interaksjoner kan ha følgende effekter:
\begin{itemize}
\item \textbf{Additiv effekt} vil si at totaleffekten av flere legemidler er lik summen av effekten av de enkelte legemidlene. Dette er den mest vanlige effekten av å ta flere legemidler sammen. 
\item \textbf{Synergisk effekt} oppstår når totaleffekten av å ta flere legemidler er større enn summen av effekten av de enkelte legemidlene. 
\item \textbf{Antagonistisk effekt} oppstår når effekten av ett eller flere legemiddel blir redusert av et annet legemiddel.
\end{itemize}

\section{Bivirkninger}
Bivirkninger er alle ikke-tilsiktede effekter et legemiddel kan ha. De fleste bivirkninger er uønskede. Smerter, ubehag eller andre symptomer som oppstår ved bruk av legemidler kan være bivirkninger. Noen ganger kan det være vanskelig å avgjøre om symptomer skyldes bivirkninger, eller om de skyldes sykdom eller andre forhold.

Noen bivirkninger er forbigående, mens andre kan være alvorlige og i noen tillfeller også livstruende. Hvilke bivirkninger som er farlig er avhengig av sykdomsbildet til pasienten, hvilke legemidler pasienten tar og situasjonen for øvrig. 

Alle legemidler kan gi bivirkninger, men ikke alle som tar legemidler får bivirkninger. Det er vanskelig å forutsi hvilke pasienter som vil få bivirkninger. Eldre er mer utsatt for mange typer bivirkninger \citep{AdverseDrugReactions}.

Ikke alle bivirkninger av legemidler er kjent. Derfor er det nyttig at pasienter rapporterer bivirkninger de opplever ved bruk av legemidler. Bivirkninger kan rapporteres til Legemiddelverket via et meldeskjema\footnote{Legemiddelverkets meldeskjema for bivirkninger: \url{http://www.legemiddelverket.no/Bivirkninger/Meld_bivirkninger/Sider/default.aspx}}, eller ved å fortelle legen om dem.

Bivirkninger forekommer med ulik hyppighet. En bivirkning kalles:
\begin{itemize}
\item \textbf{svært vanlig} hvis den forekommer hos mer enn 1 av 10 pasienter,
\item \textbf{vanlig} hvis den forekommer hos mellom 1 av 10 og 1 av 100 pasienter,
\item \textbf{mindre vanlig} hvis den forekommer hos mellom 1 av 100 og 1 av 1\,000 pasienter,
\item \textbf{sjelden} hvis den forekommer hos mellom 1 av 1\,000 og 1 av 10\,000 pasienter,
\item \textbf{svært sjelden} hvis forekommer hos færre enn 1 av 10\,000.
\end{itemize}
