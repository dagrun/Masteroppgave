\chapter*{Abstract}
\addcontentsline{toc}{chapter}{Abstract}

The use of drugs is an important measure in the public health service, but it is often poorly coordinated and rarely placed under assembled control. The lack of coordination may result in patient drug treatments that are less than optimal for some patients. Patients with an overview of their own drug use, and access to this information on a daily basis, can achieve better premises for obtaining an optimal drug treatment. Today patients have access to multiple drug information sources, but these are not personalized and often use a language that require an understanding of medical terms.      

The overall objective of this thesis was to make personal drug information easier accessible for patient than it is today. This thesis aims to answer the following research question:
\begin{quote} Can visual representation of personal drug information provide patients with answers to questions faster, and provide them with more correct knowledge, about drug-drug interactions and side effects, than text from patient information leaflets?\end{quote}

A prototype of an interactive system containing a visual representation of personal drug information was developed. The prototype was used in an experiment with the purpose of answering the research question. The experiment compared the prototype to patient information leaflets. It exanimated if patient were able to find information faster, answer questions more correctly and obtain a higher learning outcome with the use of the prototype, then with the use of patient information leaflets.

The experiment concluded that the prototype provides patients with answers to questions faster, and provide them with more correct knowledge, about drug-drug interactions and side effects, than text from patient information leaflets.

