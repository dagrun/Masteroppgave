\chapter{Eksperimentdesign} \label{chap:eksperimentdesign}
Frem til nå har motivasjon, bakgrunn og dagens situasjon for oppgaven blitt presentert. Nå er det på tide å se på denne oppgavens bidrag til forskningen. I dette kapittelet presenteres eksperimentdesignet for oppgaven i form av en overordnet forskningsplan.

\section{Valg av forskningsmetode}
Forskningsmetoder deles inn i to grupper: Kvalitativ og kvantitativ. Ved bruk av kvalitativ metode går man i dybden på et tema ved hjelp av samtaler, intervju eller observasjon. Kvantitativ metode fokuserer på å få en bredere forståelse av et tema ved å samle data fra mange deltakere \citep{Halvorsen1993}.

Forskningsspørsmålet legger føringer på valg av metode i oppgaven:
\thesisRQ

For å kunne ta stilling til forskningsspørsmålet planlegger vi å bruke en kvalitativ metode for å sammenligne en digital prototype og pakningsvedlegg. Den digitale prototypen kalles \textit{Mine Medisiner}. Eksperimentet skal undersøke de to systemenes måte å presentere legemiddelinformasjon på. Eksperimentet består av flere kasus. I hver kasus utfører én person(deltaker) et sett med forhåndsdefinerte oppgaver ved hjelp av enten pakningsvedlegg eller prototypen. Vi skal observere deltakerne mens de utfører oppgavene for å få et inntrykk av hvordan de interagerer med det aktuelle systemet. 

Ved gjennomføring av eksperimenter for å sammenligne to systemer, bør systemene være like ``ferdige'' \citep{toftoy2011praktisk}. Det er som regel ikke gunstig å sammenligne en prototype med et ferdig system. For å få gode resultater må vi derfor påse at begge systemene blir like ferdige for de oppgavene som skal utføres i eksperimentet. Eksperimentet skal inneholde oppgaver som er mulig å løse tilfredsstillende ved bruk av både den digitale prototypen og pakningsvedlegg.

\section{Arbeidsprosessen}
Vi har følgende forskningsplan:
\begin{enumerate}
\item Gjennomføre litteraturstudium for å undersøke ulike tema innenfor problemområde (Kapittel~\ref{chap:lit}).
\item Utvikle en digital prototype: Mine Medisiner (Kapittel~\ref{chap:utvikleprototype}).
\item Utføre eksperiment på pakningsvedlegg og Mine Medisiner (Kapittel~\ref{chap:gjennomforing}).
\item Analysere resultat av eksperimentet (Kapittel~\ref{chap:resultat}).
\end{enumerate}

Arbeidet vil starte med et litteraturstudium for å undersøke problemområdet og hva som er gjort tidligere. 

Vi skal sammenligne pakningsvedlegg og Mine Medisiner sine måter å presentere informasjon. For å kunne gjøre dette må Mine Medisiner utvikles. På grunn av begrenset tid anser vi det ikke som realistisk å få utviklet en ferdig versjon av Mine Medsiner. Vi skal derfor utvikle en digital prototype vil bli hardkodet til å bare fungere for et fåtall fiktive pasienter. 

For å gjøre sammenligning av Mine Medisiner og Pakningsvedlegg vil vi utføre forsøk på begge systemene. Resultatet av disse forsøkene vil til slutt bli analysert for å undersøke hvilke påstander de underbygger.

\section{Rekruttere deltakere}
For å få best mulig resultater ønsker vi at deltakerne i størst mulig grad skal kunne relatere til oppgavene og legemidlene som blir brukt i eksperimentet. Eksperimentet kan imidlertid ikke basere seg på deltakernes virkelige legemiddelsituasjon fordi denne informasjonen er sensitiv. Planen er å rekruttere \acrshort{tia}\footnote{TIA: Hjerneslag hvor symptomene trekker seg tilbake i løpet av 24 timer}-pasienter til eksperimentet, og at oppgavene som skal gjennomføres skal baseres på en fiktiv \acrshort{tia}-pasient. Prototypen av Mine Medisiner må derfor tilpasses denne fiktive \acrshort{tia}-pasienten. 

\acrshort{tia}-pasienter kan være vanskelig å rekruttere fordi det er en snever gruppe vi har liten kjennskap til. Planen er å gjøre rekrutteringen gjennom kontakter på St. Olav.


\section{Analyseplan}
Underveis i arbeidet med masteroppgaven vil vi skrive en felles forskerlogg i et delt dokument på internett. Denne forskerloggen vil inneholde alt av tanker og idéer vi har underveis. 

Vi vil gjennomføre mange møter underveis i arbeidsprosessen. Før alle møtene vil vi utarbeide møteplaner i delte dokumenter på internett. Under møtene fylles dette dokumentet inn med referat av det som blir sagt og gjort.

For hver deltaker vil vi lage et delt dokument som inneholder viktige punkter for hva vi skal notere underveis i gjennomføringen. På den måten går det raskere å ta notater. Etter hver kasus renskrives notatene for å sikre at de er forståelige til senere bruk. Det vil også tas vare på eventuelle skjema som deltakerene fyller ut. Det vil føres på et identifiserende tall på alt materiale fra hver deltaker, for å kunne skille kasusene fra hverandre. 

Datamaterialet fra kvalitative analyser er gjerne omfattende og ustrukturert. Notater fra kasus, forskerlogg og møtereferater vil for oss utgjøre det \citep{patton2002qualitative} kaller den ufordøyde, komplekse virkelighet. For å bearbeide dette ustrukturerte materiale vil vi benytte oss av en prosess kalt \textit{åpen koding}. Ved åpen koding løsriver man seg fra forskningsspørsmålet og skriver ned alt man ser. I denne prosessen gis navn til ulike deler av datamaterialet. Disse navnene kalles koder. For eksempel kan man skrive koder i margen som forklarer hva hvert avsnitt i en tekst inneholder \citep{nilssen2012analyse}. 

Etter kodeprosessen vil vi sitte igjen med veldig mange koder. I det neste steget vil vi klassifisere kodene, ved å se på sammenhengen mellom dem og hvilke mønstre de danner. Klassifiseringene skal beskrive essensen i datamaterialet og danne basis for å skrive om resultatet av eksperimentet(kapittel~\ref{chap:resultat}). Resultatene skal brukes til å underbygge påstander (kapittel~\ref{chap:analyse}). For å vurdere hvor bastante påstander vi kan komme med vil vi se på signifikansen av resultatene de underbygges av.


\section{Oppnå gode resultater} 
Feilkilder vil alltid påvirke resultatene i et forskningsprosjekt. Feil kan oppstå i alle faser av arbeidet. Ved å være klar over potensielle feilkilder er det mulig å ta hensyn til dem, og minimere påvirkningen av feilene.

Validitet\footnote{Les mer om validitet her: \url{https://sml.snl.no/validitet}} betyr gyldighet, og måler i hvilken grad det kan trekkes gyldige slutninger ut i fra en undersøkelse. For eksempel måler validiteten av eksperimentet vårt i hvor stor grad eksperimentet gir oss svar på forskningsspørsmålet vår.

Indre validitet handler om hvorvidt resultatene er gyldige for utvalget og eksperimentet som er gjennomført. Indre validitet kan være vanskelig fordi faktorer man ikke har kontroll over kan påvirke resultatene. Den indre validiteten av resultatene kan bli svekket ved at deltakerne påvirkes av de som utfører eksperimentet. For å øke den indre validiteten av eksperimentet skal vi ha en prøvegjennomføring for å finne svakheter i oppsettet, og gjøre forbedringer før den ordentlige gjennomføringen. 

Ytre validitet handler om hvorvidt et resultat kan overføres til andre enn de som deltok på undersøkelsen. Dersom utvalget er veldig skjevt, og ikke gjenspeiler befolkingen, kan det føre til lav ytre validitet. For å sørge for at den ytre validiteten ikke blir for lav skal vi prøve å rekrutere et representativt utvalg av \acrshort{tia}-pasienter. Det at vi planlegger å bare ha deltakere som er \acrshort{tia}-pasienter gjør at eksperimentet ikke nødvendigvis blir overførbart til målgruppen vår for øvrig. 

Ytre validitet kan være en utfordring når kunstige situasjoner skapes i forsøkssammenheng. For å motvirke dette skal vi gjennomføre kasusene uten å forstyrre deltakerne unødig underveis.

Hawthorne-effekt er når noens adferden endres fordi de blir studert\footnote{Les mer om Hawthorne-effekten her: \url{https://snl.no/Hawthorneeffekten}}. I forskning er det et mål å studere noe slik det er. Dersom forskningen i seg selv påvirker resultatene vil de være ugyldige. Vi vil forsøke å ikke kommentere ting underveis i gjennomføringen av forsøkene slik at deltakerne ikke blir påminnet at de blir observert mer enn nødvendig. Ting som skal diskuteres vil derfor bli tatt opp helt til slutt. 

Reliabilitet\footnote{Les mer om reliabilitet her: \url{https://sml.snl.no/reliabilitet}} betyr pålitelighet, og måler i hvilken grad man får samme resultat dersom en undersøkelse gjentas. Vi planlegger å bare ha 10-15 deltakere. Et så lavt antall kan gi litt lav pålitelighet. For å kunne få pålitelige resultater tross det lave antallet deltakere skal vi gjennomføre eksperimentet på en liten deltakergruppe (\acrshort{tia}-pasienter). Ved å ha en liten og veldefinert gruppe med deltakere tror vi det er større sannsynlighet for at deltakerne kan bli representative for gruppen og at det derfor vil gi samme resultatet å gjøre eksperimentet på nytt.


