\chapter*{Sammendrag}
\addcontentsline{toc}{chapter}{Sammendrag}
\pagenumbering{roman} 				
\setcounter{page}{1}

Bruk av legemidler er et viktig tiltak i helsevesenet, men er ofte lite koordinert og ikke underlagt samlet kontroll. Denne mangelen på koordinering kan føre til at pasienters legemiddelbehandling ikke blir så god som den kunne vært. At pasienten selv har oversikt over egen legemiddelsituasjon, og tilgang til informasjon om denne, kan gi bedre forutsetninger for å påse at legemiddelbehandlingen blir optimal. Pasienter har tilgang til flere legemiddelinformasjonskilder, men disse er ikke personlig tilpasset og bruker ofte språk som krever forståelse av medisinske ord og uttrykk. 

Det overordnede målet med masteroppgaven var å gjøre personlig legemiddelinformasjon lettere tilgjengelig for pasienter enn i dag. Følgende forskningsspørsmål ble besvart: 
\thesisRQ

Det ble utviklet en prototype av et interaktivt system, med en visuell fremstilling av personlig legemiddelinformasjon. Formålet med prototypen var å bruke den i et eksperiment for å besvare forskningsspørsmålet. Eksperimentet sammenlignet  prototypen med pakningsvedlegg for å undersøke om bruk av prototypen gjorde pasienter i stand til å finne informasjon raskere, besvare spørsmål riktigere og oppnå større læringsutbytte enn ved bruk av pakningsvedlegg. 

Konklusjonen av eksperimentet var at prototypen gir raskere svar på spørsmål, og mer korrekt kunnskap, om bivirkninger og interaksjoner, enn tekst fra pakningsvedlegg.