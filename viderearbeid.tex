\chapter{Videre arbeid} \label{chap:videreArbeid}

Dette kapittelet presenterer mulige videreføringer av prosjektet.

I dette prosjektet ble det gjennomført en kvalitativ studie, med få deltakere, som sammenlignet Mine Medisiner med pakningsvedlegg. Forsøket som ble utført gikk i dybden, for å samle informasjon på et smalt felt som ikke er mye undersøkt. En kvalitativ studie kan følges opp med en kvantitativ undersøkelse for å belyse problemområdet fra flere vinkler. En kvantitativ undersøkelse kan etterprøve resultatene og kan styrke validiteten av det som er gjort, dersom resultatene samsvarer. En kvantitativ undersøkelse kan gi bedre forståelse av resultatene som er avdekket i den kvalitative studien ved å sette dem i større sammenheng.

Per i dag gir ingen verktøy oversikt over effekten av egne legemidler, slik Mine Medisiner er ment å gjøre. Resultatet av forsøket vårt var at prototypen for Mine Medisiner gav raskere og mer korrekt informasjon om bivirkninger og interaksjoner enn pakningsvedlegg. Fordi resultatene tyder på at legemiddelinformasjon kan formidles på en bedre måte enn i dag, kan det være hensiktsmessig å videreutvikle Mine Medisiner til et ferdig system. Systemet kan gjøres tilgjengelig for hele befolkningen på for eksempel helsenorge.no, som er den offentlige inngangsporten til helse- og omsorgstjenester på nett. 

Den ferdige versjonen av Mine Medisiner må bygge på en kunnskapsrepresentasjon for å kunne resonnere om legemidler. En kunnskapsrepresentasjon er avhengig av gode kompetansespørsmål for å definere problemområdet og vurdere hvor god kunnskapsrepresentasjonen er. Etter å ha etablert kompetansespørsmålene bør eksisterende kunnskapsrepresentasjoner for legemidler, som for eksempel DrOn, studeres. For å kunne bruke eksisterende kunnskapsrepresentasjoner må de tilpasses og fylles med norske kliniske terminologier fra for eksempel ordnett, ICD10, \acrshort{mesh}, finnkode.no og legemiddelhåndboken. Arbeidet med å fylle kunnskapsrepresentasjonen kan effektiviseres ved å benytte en automatisert prosess for å hente informasjon fra de ulike kildene. 

HAFE-prosjektet, se vedlegg~\ref{HEFE}, er et steg på veien mot å lage en kunnskapsrepresentasjon til bruk i Mine Medisiner. Ontologien i HAFE-prosjektet er god nok til at prototypen til Mine Medisiner kunne brukt den. Ontologien er begrenset til et fåtall brukere og inneholder forenklinger som er tilpasset prototypen. Disse forenklingene gjør imidlertid at ontologien ikke er i stand til å modellere virkeligheten på en tilstrekkelig måte for en ferdig versjon av Mine Medisiner. For at en ontologi skal være god nok for en ferdig versjon av Mine Medisiner må den være komplett og korrekt for hele befolkningen. En slik ontologi bør, i større grad enn i HAFE-prosjektet, baseres på DrOn og andre anerkjente prosjekter. 

