\chapter{Analyse}\label{chap:analyse}
Dette kapittelet presenterer fire påstander som underbygges av resultatene i  kapittel~\ref{chap:resultat}. 


\section[Tidsbruk]{Bruk av Mine Medisiner er raskere enn bruk av pakningsvedlegg}
Gjennomsnittstiden for å ta stilling til de syv utsagnene for deltakergruppen til Mine Medisiner var 5:32, mot 16:16 for deltakergruppen til pakningsvedlegg. Differansen er stor. Tidsbruken gir imidlertid lite informasjon når den vurderes alene fordi raskere gjennomføring kan ha ført til flere feil ved besvaring av utsagnene. Det er derfor interessant å se på tidsbruket i sammenheng med antall riktige svar når man vurderer hvor rask deltakerne er til å ta stilling til utsagnene. Total tid og tid per riktig svar er gitt i tabell~\ref{tab:tidperrettMed}. Tabellen viser at tiden per riktige svar er mye lavere for Mine Medisiner, enn for pakningsvedlegg. 

Ut ifra tabellene ser det ut til at bruken av Mine Medisiner er raskere enn bruken av pakningsvedlegg. For pakningsvedlegg varierete det mye hvor lang tid deltakerne brukte per riktige svar. Så store forskjeller i et forsøk med så få deltakere gjør det vanskelig å si noe om trender i tidsbruk.

\todo{Lage figur som viser overlapp std.avvik}

\begin{table}[H]
    \centering
    \begin{tabular}{ | p{2cm} | p{3cm} |  p{3cm} |  }
      \hline
       \textbf{Deltaker nr.} & \textbf{Tid totalt} & \textbf{Tid per riktige}\\ \hline
        1 & 10:30  &1:31\\ \hline
        2 & 10:11 &2:33\\ \hline
        4 & 13:30  &4:30\\ \hline
        5 & 26:54  &6:44\\ \hline
        6 &  11:17 & 1:53\\ \hline
        12 & 4:33 & 00:39\\ \hline
        13 & 36:51 &5:16\\ \hline \hline
        \textbf{Gj.snitt} & 16:16 & 03:18 \\ \hline
        \textbf{Std.avvik} & 11:22 & 02:13\\ \hline
    \end{tabular}
    \caption{Tid per riktige svar ved bruk av pakningsvedlegg}
    \label{tab:tidperrettPak}
\end{table}

\begin{table}[H]
    \centering
    \begin{tabular}{ | p{2cm} | p{3cm} |   p{3cm} | }
      \hline
       \textbf{Deltaker nr.} & \textbf{Tid totalt} & \textbf{Tid per riktige} \\ \hline
        3 & 03:29 & 00:30\\ \hline
        7 & 04:47 & 00:57\\ \hline
        8 & 05:39 & 00:48\\ \hline
        9 &  07:26 & 1:14\\ \hline
        10 & 03:51 & 00:39\\ \hline
        11 & 08:00 & 1:20\\ \hline \hline
        \textbf{Gj.snitt} & 5:32 & 00:55 \\ \hline
        \textbf{Std.avvik} & 1:52 & 00:20  \\ \hline
    \end{tabular}
    \caption{Tid per riktige svar ved bruk av Mine Medisiner}
    \label{tab:tidperrettMed}
\end{table}

Vi undersøkte om demografiske forskjeller påvirket hvor lang tid deltakerene brukte på å ta stilling til utsagnene. Antagelsen var at alder og utdanningsnivå ville påvirke deltakernes tidsbruk. For eksempel ved at at de over 65 kom til å bruke lenger tid, og at dette måtte tas hensyn til i analysen ved å trekke fra den ekstra tiden som kunne skyldes alder. Det var imidlertid ikke noe som tydet på at demografi påvirket hvor raske deltakerene var, og vi så derfor bort ifra demografien i analysen av tidsbruken.

Halvparten av deltakerne i gruppen til Mine Medisiner hadde utdanning innen IT. Selv om det ikke er noe i resultatene som tyder på at utdanning innen IT gjorde at deltakerne tok stilling til utsagnene raskere, kan disse deltakerne lettere ha forstått hvordan grensesnittet fungerte. Dette kan ha ført til at gjennomsnittstiden ble lavere enn den ville blitt for et mer representativt utvalg deltakere. Et utsagn fra en av deltakerne: \begin{quote} \textit{ “Det er forståelig fordi brukergrensesnittet er kjent, og ligner på mange andre grensesnitt'' } \end{quote}

Ordene som ble brukt for å beskrive bruken av pakningsvedlegg viste tydelig at opplevelsen var preget av at det tok lang tid å vurdere utsagnene. Ord som gikk igjen var \textit{tidkrevende, ineffektiv og treg}. Begrunnelsen deltakerne gav var at det var store mengder informasjon det tok lang tid å lese på grunn av vanskelig språk og manglende struktur. Et utsagn fra en av deltakerne: \begin{quote} \textit{ “Det tar tid å lese gjennom hele pakningsvedlegg. Til slutt orker man ikke å lese alt og bare skummer igjennom.'' } \end{quote}

Mine medisiner ble beskrevet som \textit{tidsbesparende og effektiv}. Dette forklarte deltakerne med at all relevant informasjon var samlet på ett sted, og at det var enkelt å få oversikt. 

\textbf{Analysen førte til påstanden: \textit{Bruk av Mine Medisiner er raskere enn bruk av pakningsvedlegg}. Tendenser i tidsbruken underbygger påstanden, men på grunn av høye standardavvik kan vi ikke påstå at det er “mye`` raskere å bruke Mine Medisiner enn pakningsvedlegg. }

\section[Riktighet]{Utsagn vurderes mer riktig ved bruk av Mine Medisiner, enn ved bruk av pakningsvedlegg}
Syv utsagn ble tatt stilling til i forsøkene. Tabell~\ref{tab:PaastanderPak} og tabell~\ref{tab:PaastanderMed} i kapittel~\ref{chap:resultat} viser andel riktige vurderinger for henholdvis pakningsvedlegg og Mine Medisiner. I gjennomsnitt vurderte deltakerene i forsøket med Mine Medisiner 88\% av utsagnene riktig, mot 78\% for deltakerne til pakningsvedlegg. Forskjellen mellom andelen riktige svar ved bruk av Mine Medisiner og pakningsvedlegg faller innenfor standardavvikene, og resultatet er dermed ikke signifikant. 

Deltakerene som brukte Mine Medisiner gjorde færre feilvurderinger enn de som brukte pakningsvedlegg. Snittet for antall feil var 2\% for Mine Medisiner, mot pakningsvedlegg med 12\% feil. Forskjellene er imidlertid ikke store nok til å falle utenfor standardavvikene. 

Det er ikke mulig å si om visualisering av informasjon direkte førte til at utsagnene ble vurdert mer riktig ved bruk av Mine Medisiner enn ved bruk av pakningsvedlegg, men tilbakemeldinger fra deltakerene kan tyde på at visualiseringen spilte en rolle. Deltakerne beskrev Mine Medisiner som intuitiv og og forståelig. Flere fremhevet at interaksjonsgrafen~\ref{fig:interaksjonsGraf} gjorde det enkelt å finne informasjon om interaksjoner: \begin{quote} \textit{ “Det var litt kult med den grafen, den var fancy. Det var enkelt å se hvordan alle legemidlene fungerte sammen.'' } \end{quote}

Som tidligere nevnt hadde halvparten av deltakerene i forsøket med Mine Medisiner utdanning innen \acrshort{it}. Innenfor denne fagdisiplinen er grafer er et vanlig verktøy for å visualisere informasjon. Et utvalg deltakere med god kjennskap til grafer kan ha ført til bedre forståelse av informasjonen som ble visualisert i en graf i Mine Medisiner, enn det som ville vært tilfelle for et mer representativt utvalg deltakere. 

Pakningsvedleggene ble oppfattet som vanskelige å bruke. Begrunnelsen var at pakningsvedleggene inneholdt mye informasjon som ikke var personlig tilpasset og at språket var tungt. Tilbakemeldinger fra deltakerene tyder på at tungt språk, mye informasjon og liten grad av personlig tilpasning kan være viktige faktorer til at pakningsvedlegg kom dårligere ut enn Mine Medisiner:


\begin{quote} \textit{ ``Nei, jeg skjønner veldig godt at jeg ikke liker pakningsvedlegg. Kanskje ville pakningsvedleggene sagt meg mer hvis jeg hadde kunne litt mer om legemidler og hvilke gruppe de tilhører. Men for den vanlige mannen i gaten er dette håpløst.'' } \end{quote}

\begin{quote} \textit{ ``Det står så mye som folk uten utdannelse innen medisin ikke forstår noe av. '' } \end{quote}

\begin{quote} \textit{ ``....menstruasjonssmerter er jo ikke så interessant for meg.'' } \end{quote}

\begin{quote} \textit{ ``Dette er veldig upersonlig. Jeg vil gjerne vite hva som gjelder for meg og ikke bare for folk generelt.'' } \end{quote}

\begin{quote} \textit{ ``Du må forstå litt legeting for å forstå pakningsvedlegget.'' } \end{quote}

\begin{quote} \textit{ ``Det er tungvindt å lese når det står så mye.'' } \end{quote}

\begin{quote} \textit{ ``Jeg klarer ikke å finne ut av interaksjoner ut i fra pakningsvedleggene. Det er så mye ord og uttrykk jeg ikke forstår.'' } \end{quote}


Utsagn vurderes mer riktig ved bruk av Mine Medisiner, enn ved bruk av pakningsvedlegg
----> Pasienter vurderer legemiddelsituasjon riktigere 

\textbf{Analysen førte til påstanden: \textit{Pasienter 
vurderer egen legemiddelsituasjon riktigere
gjør mer riktige vurderinger
forstår legemiddelinformasjon bedre
...ved bruk av Mine Medisiner, enn ved bruk av pakningsvedlegg}}


\section[Læringsutbytte]{Pasienter kan lære litt om egne legemidler både ved bruk av pakningsvedlegg og ved bruk av Mine Medisiner}
Det er mange måter å definere \textit{læring}. En form for læring er å huske informasjon slik at den kan gjengis. En annen form for læring er å oppdage ny informasjon. 

5 av 7 deltakerne i forsøket med pakningsvedlegg lærte noe utover utsagnene de skulle besvare, jf~\ref{tab:LaeringsPak}, mot 2 av 6 deltakerne med Mine Medisiner, jf~\ref{tab:LaeringsMed}. 
Selv om mange deltakere lærte noe, lærte hver deltaker veldig lite. Hver deltaker kunne få seks poeng for ny læring. Gjennomsnittet for læring av ny informasjon ved bruk av pakningsvedlegg var 0,86 poeng, tilsvarende 14\% av mulige poeng. For Mine Medisiner var gjennomsnittet 0,33 poeng, som tilsvarer 6\%. Læringen av ny informasjon var veldig lav i begge forsøken, men den var høyere ved bruk av pakningsvedlegg. 

Deltakerne i forsøket med Mine Medisiner kunne gjengi litt mer informasjon fra vurderingen av utsagnene, enn deltakerene i forsøket med pakningsvedlegg. Hver deltaker kunne få tre poeng for å gjengi informasjon, Gjennomsnittet i forsøket med pakningsvedlegg var 0,86 som tilsvarer 28\% av mulige poeng. For Mine Medisiner var snittet 1,67, som tilsvarer 56\%.

Eksperimentet tyder på at Mine Medisiner og pakningsvedlegg fører til forskjellige former for læring. Ved bruk av Mine Medisiner var deltakerenes evne til å gjengi informasjon høyere enn ved bruk av pakningsvedlegg. Deltakerenes evne til å lære ny informasjon var høyere ved bruk av pakningsvedlegg, enn ved bruk av Mine Medisiner. Generelt var læringsutbytte lav for begge systemene. Det gjaldt begge former for læring, både å gjengi informasjon og å oppdage ny informasjon.

Vi undersøkte om antall legemidler deltakerne brukte påvirket evnen til å tilegne seg kunnskap om legemidler. Antagelsen var at de deltakerne som brukte legemidler fast ville lære mer om legemidlene. Det var imidlertid ikke noe som tydet på at dette var tilfellet, og vi tok derfor ikke hensyn til antall legemidler i vurderingen av hvor mye hver deltaker lærte. 

\section[Brukbarhet]{Mine Medisiner har mye bedre brukbarhet enn pakningsvedlegg}
Poengsummen til \acrshort{sus}-skjemaene brukes til å vurdere hvor brukbart et system er. Som nevnt i delkapittel~\ref{subsec:SUS} regnes et system som brukbart hvis det har en \acrshort{sus}-poengsum på over 70, som et bedre system hvis det hat en poengsum på 75-85, og som et utmerkede systemer hvis det har en poengsum på over 85. 

\acrshort{sus}-poengsummen for pakningsvedlegg er gjengitt i tabell~\ref{tab:SUSPak}, og Mine Medisiner i tabell~\ref{tab:SUSMineMed}. Mine Medisiner har en signifikant bedre \acrshort{sus}-poengsum enn pakningsvedleggene. Mine Medisiner regnes som et utmerket system med en gjennomsnittlig poengsum på over 85, mens pakningsvedlegg har en så lav poengsum at det ikke er regnet som brukbart. 

Figur~\ref{fig:ordskyP} og figur~\ref{fig:ordskyMM} viser at deltakerne som brukte Mine Medisiner valgte mer positive ord enn deltakerne som brukte pakningsvedleggene i reaction card-delen. Dette underbygger påstanden om at Mine Medisiner er mer brukbar enn pakningsvedlegg. 

Ved å se på forklaringene deltakerne gav for ordene de valgte kan noen trender sees for hva deltakerne forklarte forskjellen i brukbarhet med.

Forklaringene deltakerene gav på valg av ord gir et inntrykk av hva som gjør Mine Medisiner mer brukbar enn pakningsvedlegg. Det ble blant annet nevnt at pakningsvedlegg var upersonlig, brukte tungt språk og at de inneholdt for mye informasjon. Deltakerne forklarte at Mine Medisiner var logisk oppbygd, og at det var lett å forstå informasjonen som ble presentert. Flere av deltakerene trakk frem at de synes en grafisk fremstilling av interaksjoner gjorde det enkelt å få oversikt. 