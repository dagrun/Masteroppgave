\chapter{Diskusjon} \label{chap:diskusjon}
Forrige kapittel presenterte analysen av resultatene for forskningen. I dette kapittelet diskuteres valgene vi har tatt, hvilke utfordringer vi har møtt underveis og gyldighetene av resultatene. 

\todo{diskutere demografi}

\section{Gjennomføringen av eksperimentet}
I de neste avsnittene diskuteres valgene vi har tatt, og hvilke alternative fremgangsmåter som kunne vært benyttet. Utfordringer som har oppstått underveis og hvordan vi har valgt å løse dem vil også bli diskutert her. 

Eksperimentet kunne blitt mer reellt ved å sammenligne et "ferdig" system med pakningsvedlegg, istedenfor en prototype. Det kunne blitt utført et forsøk i reelle situasjoner på et ferdig system, hvor deltakerene var faktiske brukere. Brukerene kunne fått tilgang til systemet hjemme, lagt inn sin egen legemiddelliste, og brukt systemet i hverdagen. Vi kunne samlet informasjon om bruk av pakningsvedlegg og Mine Medisiner over tid. På den måten kunne brukerene benyttet systemene i mer realistiske brukskontekster, og vi ville da også øke validiteten av resultatene våre.  

Begrensinger i tid og kapasitet gjorde det imidlertid ikke gjennomførbart å utvikle et ferdig system. For å i størst mulig grad ligne et ferdig system ble det utviklet en digital prototype til eksperimentet. 

På grunn av føringene forskningsspørsmålet gav ble det valgt å gjennomføre et forsøk som sammenlignet en digital prototype og pakningsvedlegg. Forskningsspørsmålet kunne vært formulert annerledes. Ved å fokusere mer på å sammenligne graf med tekst istedenfor å sammenligne Mine Medisiner med pakningsvedlegg, kunne noen variabler i forsøket antageligvis blitt eliminert. 

Ved å gjennomføre forsøkene kun på \acrshort{tia}\footnote{TIA: Hjerneslag hvor symptomene trekker seg tilbake i løpet av 24 timer}-pasienter kunne flere variabler blitt eliminert: Alder hadde blitt jevnere, forkunnskap og kjennskap til legemiddellisten hadde vært likere og deltakerne hadde hatt en bedre forståelse av konteksten. Det ville økt reliabilitet i eksperimentet vårt. Dessverre var det ikke tid eller kapasitet til å rekruttere fra en så smal brukergruppe. 

Eksperimentet utført i dette prosjektet har sammenlignet to svært forskjellige systemer. For å avklare om interaksjonsgraf og symptomsøk er bedre enn tekst, kunne vi laget en tekstbeskrivelse av den samme informasjonen som er tilgjengelig i Mine Medisiner, og sammenligne denne tekstbeskrivelsen med Mine Medisiner. Da vil forskjellen mellom systemene være mindre, og derfor vil det være færre variabler å ta hensyn til når resultatene skal analyseres. Det ville da ikke vært en sammenligning av pakningsvedlegg og Mine Medisiner, men en sammenligning av måte å presentere informasjon på. Ved å endre pakningsvedleggene på denne måten ville vi ikke fått sammenlignet Mine Medisiner med et eksisterende system, for å vurdere hvor godt det fungerer i forhold til noe pasienter har tilgang til i dag. 

Vi ønsket å knytte eksperimentet til virkeligheten ved å sammenligne en eksisterende kilde til informasjon, med en ny presentasjonsmåte. Dette var for å vurdere om informasjon kan presenteres på en bedre måten enn i dag. Denne sammenligningen ville ikke være reell, dersom relevante utsnitt av pakningsvedleggene hadde blitt konstruert for dette eksperimentet.

Systemene som ble brukt i eksperimentet var svært forskjellige. Pakningsvedlegg er et ferdig system på papirformat, mens Mine Medisiner er en digital prototype. Når to så forskjellige systemer sammenlignes er det vanskelig å vurdere hva som forårsaker resultatene. Det er mange ting som skiller systemene, og som dermed kan føre til at systemene presterer ulikt. 
Hver deltaker fikk bare utføre forsøket på ett av systemene. Dette gjør det vanskeligere å sammenligne systemene fordi ulikheter mellom deltakere kan ha bidratt til ulikhetene i resultatene. At hver deltaker ikke utfører forsøket på begge systemene hindret at deltakerene brukte det de lærte ved å bruke det ene systemet når de skulle bruke det neste. En annen måte å hindre dette på er ved å gjennomføre forsøk på begge systemene for alle deltakerene, men ha forskjellige oppgaver på de to systemene. Dette fører imidlertid til flere usikkerhetsmomenter i vurderingen fordi det er vanskelig å lage oppgaver som går ut på de sammen tingene med samme vanskelighetsgrad.

Forsøkene tok ca en time å gjennomføre, og det var mange variabler som påvirket forsøkene. At forsøkene var så omfattende gjorde at det bare var mulig å gjennomføre 13 forsøk totalt. Dersom hvert forsøk hadde vært mindre omfattende kunne det blitt gjennomført flere forsøk, og det hadde vært færre variabler å ta hensyn til under analysen av resultatene. Gjennomføring av flere forsøk ville kunne gitt bedre grunnlag for statistisk analyse, men kunne gått på bekostning av innsikt i den enkelte deltakers vurderinger. 


\section{Gyldighet av resultater}
Det er viktig å ta høyde for eventuelle feilkilder som kan ha ført til ugyldige resultater. I de neste avsnittene diskuteres styrken på resultatene, og hvilke faktorer som styrker eller svekker påliteligheten av dem. 

Det ble gjennomført henholdsvis 6 og 7 forsøk av hvert system. Utvalget er for lite til å gjøre statistisk analyse av resultatene. For å motvirke at tilfeldigheter påvirket resultatene kunne det vært gjennomført flere forsøk. Mye tyder på at analysen stemmer fordi det er et mønster i resultatet til forsøkene som er gjennomført, men det lave antallet forsøk gjør at grunnlaget for analysen var tynt. 

Utdanningsnivået til deltakerne er ikke representativt for befolkningen. 12 av 13 av deltakerne hadde høyere utdanning, mot ca 50 \% av befolkningen generelt. Dette gjorde at den ytre validiteten til forsøket ble svekket. Det høye utdanningsnivået har påvirket begge gruppene, fordi snittet av utdanningsnivå i de to gruppene var relativt likt. 

Fordi Mine Medisiner er et datasystem, kan forsøket hvor deltakerene fikk bruke Mine Medisiner ha blitt påvirket i positiv retning av at halvparten av deltakerne hadde utdanning innen \acrshort{it}. I gruppen som utførte oppgaver ved hjelp av pakningvedlegg hadde også halvparten utdanning innen \acrshort{it}, men dette har nok ikke påvirket resultatene i like stor grad. 

Det var store demografiske forskjeller mellom deltakerne, noe som fører til ekstra variabler som kan påvirke resultatet. På grunn av at demografien innad i hver gruppe er relativt lik, er de demografiske forskjellene sett bort i fra i analysen. Det er mulig at demografiske fakta som ikke ble avdekket kan ha påvirket resultatene. 

Pakningsvedlegg var kjent for fleste. Deltakerenes tidligere erfaring med pakningsvedlegg kan ha påvirket svarene de gav, og hvordan de vurderte systemet. 

Vi kan ikke utelukke metodiske svakheter, eller at det finnes forskning som viser til andre resultater enn de vi har funnet.