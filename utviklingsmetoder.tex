\chapter{Utviklingsmetoder} \label{chap:utviklingsmetoder}

De forrige kapittelene presenterte en plan for forskningen og knyttet dette opp mot hva som allerede er gjort, gjennom et litteraturstudie. Dette kapittelet handler om utviklingsmetodene benyttet for å lage en prototype av Mine Medisiner. Med forskningsspørsmålet i tankene, var det åpenbart at vi måtte rette utviklingen mot pasienter. Vi valgte en brukersentrert utviklingsprosess, jf. delkapittel~\ref{sec:brukersentrert}. Dette er en utviklingsfilosofi som aktivt involverer brukere gjennom utviklingsprosessen. 

En oversikt over hvilke utviklingsmetoder vi har brukt, og hensikten med de ulike metodene er vist i tabell \ref{tab:utviklingsmetodene}.

\begin{table}[H]
    \centering
    \begin{tabular}{ | l | p{7cm} | }
      \hline
      \textbf{Metode} & \textbf{Hensikt} \\ \hline
      Personae & Bli kjent med målgruppen \\ \hline
      User Stories & Fortså brukeren og utforske samspillet mellom brukeren og systemet \\ \hline
      Workshop & Utveklse idéer og meninger \\ \hline
      Semistrukturert intervju & Samle informasjon og forstå legemiddelfagfeltet \\ \hline
      Prototyping & Teste designmuligheter og få tilbakemeldinger som gjorde det mulig å lage en bedre prototype \\ 
      \hline
    \end{tabular}
    \caption{Oversikt over utviklingsmetodene som ble brukt}
    \label{tab:utviklingsmetodene}
\end{table}


\section{Personae}
Personae er personprofiler utviklet for å beskrive målgruppen til en tjeneste. Personae er et nyttig virkemiddel for å bli kjent med målgruppen og for å se for seg konkrete brukseksempler. 
\section{User Stories}
User stories beskriver hvordan brukere ønsker å interagere med et system. Med utgangspunkt i personae beskrives tjenesten fra brukerenes perspektiv. Formålet med user stories er å forstå hvordan brukerene møter systemet og å verifisere at brukerenes behov er tatt hensyn til.

User stories kan presenteres på flere måter, blant annet som story boards og tekstbeskrivelser. Et story board illustrerer en situasjon hvor et system blir benyttet. Den vanligste måten å lage story boards på er i form av tegneseriestriper. Story boards bør lages slik at hvem som helst kan forstå dem. 

\section{Workshop}
En workshop er en arbeidsform med vekt på samhandling mellom deltakerne. Deltakerne utveksler meninger og ideer for å komme frem til et felles resultat. Dette resultatet kan være abstrakt (f.eks. økt innsikt i et tema) eller konkret (f.eks. en prototype).

En workshop kan bestå av flere deler, blant annet oppgaver eller diskusjoner. Det kan variere hvorvidt delene skal utføres individuelt av hver enkelt deltaker, eller om de skal utføres sammen av gruppen. En workshop kan tilpasses underveis.

\section{Semistrukturert intervju}
Et intervju er en samtale mellom to eller flere personer, hvor en intervjuer stiller spørsmål, og et intervjuobjekt svarer. Målet med et intervju er å samle informasjon. Mye av informasjonen får man gjennom hva intervjuobjektet sier, men det er også verdifull informasjon i intervjuobjektets kroppsspråk og stemmebruk. 

Et semistrukturert intervju har gjerne en overordnet plan med samtaleemner, men samtalen drives videre av interessante tema som dukker opp underveis i intervju. En fordel med semistrukturerte intervjuer er at de kan tilpasses underveis ved å be om forklaringer, stille oppfølgingsspørsmål eller utforske nye emner som dukker opp. Denne typen tilpasning kan bidra til å forstå intervjuobjektet bedre og til å oppmuntre intervjuobjektet. Det er viktig å finne en balansegang da for mange oppfølgingsspørsmål kan gjør at intervjuobjektet føler seg utilpass \citep{Seidman}.

Bruk av personprofiler i et intervju kan være et nyttig hjelpemiddel for å unngå personlige spørsmål. Istedenfor å stille spørsmål knyttet til intervjuobjektets situasjon knytter man spørsmålene opp til personprofiler spesielt utviklet for dette formålet.

\section{Prototyping}
Prototyping er brukt som utviklingsmetode, og er presentert i delkapittel~\ref{sec:prototyping}.